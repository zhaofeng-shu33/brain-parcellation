\documentclass{article}
\def\code#1{\texttt{#1}}
\usepackage{hyperref}
\title{brain parcellation with clustering method}
\begin{document}
\maketitle

\section{Data description}
This is a description summary of fMRI data, provided by \textit{Yale 
Magnetic Resonance Research Center}.

There are 423 \code{tr} data samples, 202 \code{tb} data samples.

For each data sample, it can be treated as a 4d array,
there is
$(x, y, z, t)$, where $x = 61, y= 73, z = 61$, and  $t= 212$ ( for
 file starts with \texttt{tr}) or $217$ ( for file starts with \texttt{tb}
)
However, not every voxel is within the brain region. we use another \code{mask} array to indicate the
voxel within the valid region. The total number of valid voxel is 47763.
All the 3d images share the same mask.

After compression, each data sample uses about \code{44.8MB} storage.

In each matlab data files, there are four parameters : \code{img, simg, mask ,pos}.
\section{Method}
If only image data of only one subject is given, first convert the 4d array data to \code{(n\_samples, n\_features)}
with the image mask. And then we can apply standard clustering algorithm.
\section{Tools}
\begin{itemize}
\item \texttt{Python: nilearn}
\item \texttt{Python: nilabel}
\end{itemize}

\begin{thebibliography}{9}
\bibitem{nilearnClusteringExample} \href{https://nilearn.github.io/connectivity/parcellating.html}{nilearn parcellation example}
\bibitem{groupwise} Groupwise whole-brain parcellation from resting-state fMRI
data for network node identification
\end{thebibliography}
\end{document}
