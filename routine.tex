\documentclass{article}
\usepackage{mathtools}
\DeclarePairedDelimiter\abs{\lvert}{\rvert}
\DeclarePairedDelimiter\norm{\lVert}{\rVert}
\title{How to apply info-clustering algorithm on the brain parcellation problem}

\begin{document}
\maketitle
\begin{enumerate}
\item data loader, we don't differentiate whether the brain fMRI is from the same person.

In this step, we divide the brain cube into different blocks. For each block, we apply the reshape operation (from 4-d array to 1-d array). Then collecting different instances of 1-d array, we apply \textbf{pca} dimension reduction to reduce the dimension of features.

\item connectivity matrix computation(neural network)

For every two distinct nodes, run the \texttt{ace\_nn } routine to compute the approximation of $
\frac{1}{2}\norm{B}_F^2$

\item apply info-clustering algorithm to find suitable partition result.

For given threshold clustering number, we can find the corresponding paritition.

\item result visualization

use three-view drawing to visualize the 3D structure of human brain and  use different colors to represent different cluster.

\end{enumerate}
\end{document}